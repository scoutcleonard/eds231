% Options for packages loaded elsewhere
\PassOptionsToPackage{unicode}{hyperref}
\PassOptionsToPackage{hyphens}{url}
%
\documentclass[
]{article}
\usepackage{amsmath,amssymb}
\usepackage{lmodern}
\usepackage{iftex}
\ifPDFTeX
  \usepackage[T1]{fontenc}
  \usepackage[utf8]{inputenc}
  \usepackage{textcomp} % provide euro and other symbols
\else % if luatex or xetex
  \usepackage{unicode-math}
  \defaultfontfeatures{Scale=MatchLowercase}
  \defaultfontfeatures[\rmfamily]{Ligatures=TeX,Scale=1}
\fi
% Use upquote if available, for straight quotes in verbatim environments
\IfFileExists{upquote.sty}{\usepackage{upquote}}{}
\IfFileExists{microtype.sty}{% use microtype if available
  \usepackage[]{microtype}
  \UseMicrotypeSet[protrusion]{basicmath} % disable protrusion for tt fonts
}{}
\makeatletter
\@ifundefined{KOMAClassName}{% if non-KOMA class
  \IfFileExists{parskip.sty}{%
    \usepackage{parskip}
  }{% else
    \setlength{\parindent}{0pt}
    \setlength{\parskip}{6pt plus 2pt minus 1pt}}
}{% if KOMA class
  \KOMAoptions{parskip=half}}
\makeatother
\usepackage{xcolor}
\IfFileExists{xurl.sty}{\usepackage{xurl}}{} % add URL line breaks if available
\IfFileExists{bookmark.sty}{\usepackage{bookmark}}{\usepackage{hyperref}}
\hypersetup{
  pdftitle={EDS 231: Assignment 3},
  pdfauthor={Scout Leonard},
  hidelinks,
  pdfcreator={LaTeX via pandoc}}
\urlstyle{same} % disable monospaced font for URLs
\usepackage[margin=1in]{geometry}
\usepackage{color}
\usepackage{fancyvrb}
\newcommand{\VerbBar}{|}
\newcommand{\VERB}{\Verb[commandchars=\\\{\}]}
\DefineVerbatimEnvironment{Highlighting}{Verbatim}{commandchars=\\\{\}}
% Add ',fontsize=\small' for more characters per line
\usepackage{framed}
\definecolor{shadecolor}{RGB}{248,248,248}
\newenvironment{Shaded}{\begin{snugshade}}{\end{snugshade}}
\newcommand{\AlertTok}[1]{\textcolor[rgb]{0.94,0.16,0.16}{#1}}
\newcommand{\AnnotationTok}[1]{\textcolor[rgb]{0.56,0.35,0.01}{\textbf{\textit{#1}}}}
\newcommand{\AttributeTok}[1]{\textcolor[rgb]{0.77,0.63,0.00}{#1}}
\newcommand{\BaseNTok}[1]{\textcolor[rgb]{0.00,0.00,0.81}{#1}}
\newcommand{\BuiltInTok}[1]{#1}
\newcommand{\CharTok}[1]{\textcolor[rgb]{0.31,0.60,0.02}{#1}}
\newcommand{\CommentTok}[1]{\textcolor[rgb]{0.56,0.35,0.01}{\textit{#1}}}
\newcommand{\CommentVarTok}[1]{\textcolor[rgb]{0.56,0.35,0.01}{\textbf{\textit{#1}}}}
\newcommand{\ConstantTok}[1]{\textcolor[rgb]{0.00,0.00,0.00}{#1}}
\newcommand{\ControlFlowTok}[1]{\textcolor[rgb]{0.13,0.29,0.53}{\textbf{#1}}}
\newcommand{\DataTypeTok}[1]{\textcolor[rgb]{0.13,0.29,0.53}{#1}}
\newcommand{\DecValTok}[1]{\textcolor[rgb]{0.00,0.00,0.81}{#1}}
\newcommand{\DocumentationTok}[1]{\textcolor[rgb]{0.56,0.35,0.01}{\textbf{\textit{#1}}}}
\newcommand{\ErrorTok}[1]{\textcolor[rgb]{0.64,0.00,0.00}{\textbf{#1}}}
\newcommand{\ExtensionTok}[1]{#1}
\newcommand{\FloatTok}[1]{\textcolor[rgb]{0.00,0.00,0.81}{#1}}
\newcommand{\FunctionTok}[1]{\textcolor[rgb]{0.00,0.00,0.00}{#1}}
\newcommand{\ImportTok}[1]{#1}
\newcommand{\InformationTok}[1]{\textcolor[rgb]{0.56,0.35,0.01}{\textbf{\textit{#1}}}}
\newcommand{\KeywordTok}[1]{\textcolor[rgb]{0.13,0.29,0.53}{\textbf{#1}}}
\newcommand{\NormalTok}[1]{#1}
\newcommand{\OperatorTok}[1]{\textcolor[rgb]{0.81,0.36,0.00}{\textbf{#1}}}
\newcommand{\OtherTok}[1]{\textcolor[rgb]{0.56,0.35,0.01}{#1}}
\newcommand{\PreprocessorTok}[1]{\textcolor[rgb]{0.56,0.35,0.01}{\textit{#1}}}
\newcommand{\RegionMarkerTok}[1]{#1}
\newcommand{\SpecialCharTok}[1]{\textcolor[rgb]{0.00,0.00,0.00}{#1}}
\newcommand{\SpecialStringTok}[1]{\textcolor[rgb]{0.31,0.60,0.02}{#1}}
\newcommand{\StringTok}[1]{\textcolor[rgb]{0.31,0.60,0.02}{#1}}
\newcommand{\VariableTok}[1]{\textcolor[rgb]{0.00,0.00,0.00}{#1}}
\newcommand{\VerbatimStringTok}[1]{\textcolor[rgb]{0.31,0.60,0.02}{#1}}
\newcommand{\WarningTok}[1]{\textcolor[rgb]{0.56,0.35,0.01}{\textbf{\textit{#1}}}}
\usepackage{longtable,booktabs,array}
\usepackage{calc} % for calculating minipage widths
% Correct order of tables after \paragraph or \subparagraph
\usepackage{etoolbox}
\makeatletter
\patchcmd\longtable{\par}{\if@noskipsec\mbox{}\fi\par}{}{}
\makeatother
% Allow footnotes in longtable head/foot
\IfFileExists{footnotehyper.sty}{\usepackage{footnotehyper}}{\usepackage{footnote}}
\makesavenoteenv{longtable}
\usepackage{graphicx}
\makeatletter
\def\maxwidth{\ifdim\Gin@nat@width>\linewidth\linewidth\else\Gin@nat@width\fi}
\def\maxheight{\ifdim\Gin@nat@height>\textheight\textheight\else\Gin@nat@height\fi}
\makeatother
% Scale images if necessary, so that they will not overflow the page
% margins by default, and it is still possible to overwrite the defaults
% using explicit options in \includegraphics[width, height, ...]{}
\setkeys{Gin}{width=\maxwidth,height=\maxheight,keepaspectratio}
% Set default figure placement to htbp
\makeatletter
\def\fps@figure{htbp}
\makeatother
\setlength{\emergencystretch}{3em} % prevent overfull lines
\providecommand{\tightlist}{%
  \setlength{\itemsep}{0pt}\setlength{\parskip}{0pt}}
\setcounter{secnumdepth}{-\maxdimen} % remove section numbering
\setlength{\parindent}{1em}
\usepackage{float}
\ifLuaTeX
  \usepackage{selnolig}  % disable illegal ligatures
\fi

\title{EDS 231: Assignment 3}
\author{Scout Leonard}
\date{04/22/2022}

\begin{document}
\maketitle

\hypertarget{load-libraries}{%
\section{Load Libraries}\label{load-libraries}}

\begin{Shaded}
\begin{Highlighting}[]
\NormalTok{packages}\OtherTok{=}\FunctionTok{c}\NormalTok{(}\StringTok{"quanteda.sentiment"}\NormalTok{,}
           \StringTok{"quanteda.textstats"}\NormalTok{,}
           \StringTok{"tidyverse"}\NormalTok{,}
           \StringTok{"tidytext"}\NormalTok{,}
           \StringTok{"lubridate"}\NormalTok{,}
           \StringTok{"here"}\NormalTok{,}
           \StringTok{"wordcloud"}\NormalTok{, }\CommentTok{\#visualization of common words in the data set}
           \StringTok{"reshape2"}\NormalTok{,}
           \StringTok{"quanteda"}\NormalTok{,}
           \StringTok{"knitr"}\NormalTok{) }\CommentTok{\#devtools::install\_github("quanteda/quanteda.sentiment") \#not available currently through CRAN}

\ControlFlowTok{for}\NormalTok{ (i }\ControlFlowTok{in}\NormalTok{ packages) \{}
  \ControlFlowTok{if}\NormalTok{ (}\FunctionTok{require}\NormalTok{(i,}\AttributeTok{character.only=}\ConstantTok{TRUE}\NormalTok{)}\SpecialCharTok{==}\ConstantTok{FALSE}\NormalTok{) \{}
    \FunctionTok{install.packages}\NormalTok{(i,}\AttributeTok{repos=}\StringTok{\textquotesingle{}http://cran.us.r{-}project.org\textquotesingle{}}\NormalTok{)}
\NormalTok{  \}}
  \ControlFlowTok{else}\NormalTok{ \{}
    \FunctionTok{require}\NormalTok{(i,}\AttributeTok{character.only=}\ConstantTok{TRUE}\NormalTok{)}
\NormalTok{  \}}
\NormalTok{\}}
\end{Highlighting}
\end{Shaded}

\hypertarget{import-data}{%
\section{Import Data}\label{import-data}}

\begin{Shaded}
\begin{Highlighting}[]
\NormalTok{raw\_tweets }\OtherTok{\textless{}{-}} \FunctionTok{read.csv}\NormalTok{(}\StringTok{"https://raw.githubusercontent.com/MaRo406/EDS\_231{-}text{-}sentiment/main/dat/IPCC\_tweets\_April1{-}10\_sample.csv"}\NormalTok{, }\AttributeTok{header=}\ConstantTok{TRUE}\NormalTok{)}

\NormalTok{dat}\OtherTok{\textless{}{-}}\NormalTok{ raw\_tweets[,}\FunctionTok{c}\NormalTok{(}\DecValTok{5}\NormalTok{,}\DecValTok{7}\NormalTok{)] }\CommentTok{\# Extract Date and Title fields}

\NormalTok{tweets }\OtherTok{\textless{}{-}} \FunctionTok{tibble}\NormalTok{(}\AttributeTok{text =}\NormalTok{ dat}\SpecialCharTok{$}\NormalTok{Title,}
                  \AttributeTok{id =} \FunctionTok{seq}\NormalTok{(}\DecValTok{1}\SpecialCharTok{:}\FunctionTok{length}\NormalTok{(dat}\SpecialCharTok{$}\NormalTok{Title)),}
                 \AttributeTok{date =} \FunctionTok{as.Date}\NormalTok{(dat}\SpecialCharTok{$}\NormalTok{Date,}\StringTok{\textquotesingle{}\%m/\%d/\%y\textquotesingle{}}\NormalTok{))}


\FunctionTok{head}\NormalTok{(tweets}\SpecialCharTok{$}\NormalTok{text, }\AttributeTok{n =} \DecValTok{5}\NormalTok{) }\SpecialCharTok{\%\textgreater{}\%} 
  \FunctionTok{kable}\NormalTok{()}
\end{Highlighting}
\end{Shaded}

\begin{longtable}[]{@{}
  >{\raggedright\arraybackslash}p{(\columnwidth - 0\tabcolsep) * \real{1.0000}}@{}}
\toprule
\begin{minipage}[b]{\linewidth}\raggedright
x
\end{minipage} \\
\midrule
\endhead
thank you, followers, for the great photo suggestions for our upcoming
IPCC report - on Monday you will find the lucky one selected for our
cover from among your submissions! \\
\bottomrule
\end{longtable}

we now need a good picture on \#biofuels . any suggestions please for
which we can get copyrights fast? \textbar{} \textbar Greenpeace: The
real solution to the climate crisis will require a rapid transition away
from fossil fuels.

What else we expect from the upcoming \#IPCC report on climate
solutions, set for publication on Monday, 4 April ⬇️
\url{https://t.co/EC6a25S7tY} \textbar{} \textbar Governments have a
responsibility to ensure that \#IPCCReport is grounded in rapid phaseout
of fossil fuel use and production --- not \#FalseClimateSolutions.

Read more in our open letter: \url{https://t.co/4larBPgeba}
\url{https://t.co/Fv1OphPmac} \textbar{} \textbar Next week, the IPCC
will publish a new report detailing their new models and policy
pathways.

Want to study up before the headlines? Read @bertrandhb's second long
read on CCS, explaining how and why IPCC models use so much saviour
tech.

\url{https://t.co/6yBf0j7UWA} \textbar{} \textbar Live stream of virtual
IPCC press conference releasing the report on mitigation of climate
change, 9 a.m. GMT o\ldots{} \url{https://t.co/IqRCvvQxyX} \textbar{}

\newpage

You will use the tweet data from class today for each part of the
following assignment.

\hypertarget{part-1}{%
\section{Part 1}\label{part-1}}

Think about how to further clean a twitter data set. Let's assume that
the mentions of twitter accounts is not useful to us. Remove them from
the text field of the tweets tibble.

\hypertarget{remove-mentions}{%
\subsection{Remove mentions}\label{remove-mentions}}

\newpage

\hypertarget{part-2}{%
\section{Part 2}\label{part-2}}

Compare the ten most common terms in the tweets per day. Do you notice
anything interesting?

\newpage

\hypertarget{part-3}{%
\section{Part 3}\label{part-3}}

Adjust the wordcloud in the ``wordcloud'' chunk by coloring the positive
and negative words so they are identifiable.

\newpage

\hypertarget{part-4}{%
\section{Part 4}\label{part-4}}

Let's say we are interested in the most prominent entities in the
Twitter discussion. Which are the top 10 most tagged accounts in the
data set. Hint: the ``explore\_hashtags'' chunk is a good starting
point.

\newpage

\hypertarget{part-5}{%
\section{Part 5}\label{part-5}}

The Twitter data download comes with a variable called ``Sentiment''
that must be calculated by Brandwatch. Use your own method to assign
each tweet a polarity score (Positive, Negative, Neutral) and compare
your classification to Brandwatch's (hint: you'll need to revisit the
``raw\_tweets'' data frame).

\end{document}
